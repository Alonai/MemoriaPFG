\chapter{Manual de Usuario}

	\section{Menú principal}

		Al ejecutar la aplicación, se le muestra al usuario un menú principal. Dicho menú cuenta con tres opciones:

		\begin{itemize}

			\item Jugar

				Esta opción dará comienzo a una partida. El jugador será traslado a la pantalla de juego, donde se desarrollará la partida.

			\item Créditos

				Esta opción mostrará la pantalla de créditos, la cual contará con un botón \textit{Atrás}, el cual devolverá al usuario al menú principal.

			\item Salir

				Esta opción cerrará la aplicación.

		\end{itemize}

	\section{Objetivo}

		Una vez dentro de la partida, existen dos formas de finalizarla:

		\begin{itemize}		

			\item \textbf{Eliminando a todos los enemigos}: El jugador, valiéndose de las tres habilidades de las que dispone, debe acabar con todos los enemigos que hay en el nivel. Si lo consigue, la partida terminará y el jugador será enviado de vuelta al menú principal automáticamente.

			\item \textbf{Excediendo el tiempo de partida}: Alternativamente, si el jugador pasa demasiado tiempo sin conseguir eliminar a todos los enemigos del nivel y la cuenta atrás mostrada en pantalla llega a cero, la partida finalizará automáticamente, devolviendo al jugador al menú principal.

		\end{itemize}

	\section{Controles}

		El jugador, para interactuar con el juego, debe utilizar el siguiente esquema de controles:

		\begin{itemize}

			\item En el menú principal, el usuario debe interactuar con los botones mediante el ratón, es decir, para seleccionar cualquier de las opciones, debe hacer click en el botón correspondiente.

			\item Una vez en partida, el resto de las entradas se realizan mediante el teclado. Para moverse por el escenario, el jugador debe utilizar las teclas direccionales. Para utilizar sus habilidades, deberá utilizar las teclas \textit{A}, \textit{S} o \textit{D} en función de la que quiera ejecutar.

			\begin{itemize}

				\item \textbf{A}: Disparar flecha.

				\item \textbf{S}: Colocar bomba.

				\item \textbf{D}: Ataque cuerpo a cuerpo.

			\end{itemize}

		\end{itemize}