\chapter{Conclusiones y Líneas Futuras}

\section{Visión general}

	Este capítulo recoge, a modo de conclusión, la visión del equipo de trabajo tras la realización del proyecto, expresando sus opiniones acerca del trabajo realizado teniendo en cuenta contribuciones personales y desarrollo profesional.

\section{Objetivos cumplidos}

	A pesar de no contar con tanto tiempo como se hubiese deseado para la realización del proyecto debido a diversas responsabilidades, se podría considerar que los objetivos del proyecto han sido cumplidos, tal como se definen en el capítulo correspondiente.

	\begin{itemize}

		\item Se ha desarrollado un producto jugable y estable.

		\item Las características principales están implementadas.

		\item La arquitectura del software está preparada para añadir y modificar con facilidad nuevas funciones.

		\item La investigación hecha para desarrollar del proyecto ha permitido aprender mucho sobre el desarrollo de soluciones software de este tipo.

		\item Se han aprendido técnicas y patrones de diseño utilizados comúnmente en la industria, y se han implementado algunos de ellos.

		\item El producto está en una fase que, si se añadiesen más funcionalidades y otras características de juegos, podría comercializarse.

	\end{itemize}

	Por otro lado, los requisitos definidos del software han sido cumplidos también:

	\begin{itemize}
			\item
				
			El nivel es generado proceduralmente en cada sesión y es poblado de enemigos.
			
			\item 
				
			El juego es capaz de detectar cuando distintas entidades colisionan entre ellas o con el escenario y actúa en consecuencia.

			\item 
				
			La cámara del juego sigue al jugador en los ejes X e Y.

			\item 
				
			El juego detecta las entradas del jugador, comprueba el estado actual del mismo y actúa en consecuencia.

			\item 
				
			El juego detecta cuando se han cumplido las condiciones para fin de partida y la termina.

			\item 
				
			Cada enemigo tiene una inteligencia artificial distinta.

			\item 
				
			El jugador tiene a su disposición tres tipos de ataque.

			\item 
				
			El juego cuenta con una pantalla para mostrar los controles.

			\item
					
			Durante las partidas el juego muestra en todo momento los datos relevantes al usuario mediante una interfaz gráfica.

			\item
				
			La interfaz gráfica no molesta al jugador.

			\item
				
			El juego se controla únicamente con el teclado.

			\item 
				
			El código del juego es multiplataforma, de forma que puede compilarse en Windows, Linux y OSX.

			\item 
				
			La versión final incluye las librerías necesarias para el correcto funcionamiento del juego.

			\item
					
			El juego funciona a un mínimo de 30 \acrshort{fps} en cualquier plataforma.

			\item
				
			Durante las partidas, no hay errores que provoquen un final inesperado de la aplicación.
	\end{itemize}

\section{Consideraciones del trabajo realizado}

	Este proyecto nació de la afición a los videojuegos y del deseo a crearlos. Desde siempre el alumno ha sido un gran aficionado a jugar a estos producto, y a medida que jugaba se iba interesando más y más por ellos. Ya desde pequeño, tras pasar las épocas de astronauta y bombero, el alumno decía que se quería dedicar a ello, y con el paso de los años esa decisión no hizo más que reafirmarse. Existen muchas maneras de entrar a formar parte del mundo del desarrollo de videojuegos, pero todos los profesionales coinciden en que, a pesar de que existen ofertas formativas muy buenas, sin duda la mejor manera es simplemente haciendo juegos. Con esto en mente, el alumno se lanzó a desarrollar el proyecto con la esperanza de que fuese la mejor experiencia educativa posible.

	Para desarrollar el proyecto, el alumno tuvo que considerar una serie de aspectos. El primero fue el lenguaje de programación, ya que sería la herramienta principal del desarrollo, y la elección del mismo tendría gran repercusión. Dado que C++ es el lenguaje utilizado profesionalmente para el desarrollo de juegos, el alumno se decantó por el mismo, ya que aspira a ser un profesional del área. Además, en el grado obtuvo unas nociones básicas del mismo, de forma que la curva de aprendizaje se vería aligerada. Tras esto, comparó distintos motores y frameworks, y finalmente se decantó por \acrshort{sfml} ya que está escrito en C++ y permitía al alumno desarrollar el software a su antojo. El alumno no tenía ninguna experiencia previa con \acrshort{sfml}, de forma que tuvo que pasar por una fase de aprendizaje.

	Para asimilar el aprendizaje teórico, el alumno desarrolló unos cuantos programas pequeños para hacer con \acrshort{sfml}, de esta forma se familiarizó con la librería y fue capaz de detectar errores e inconveniencias. Para solucionarlos y mejorar su uso de la librería, se sirvió de la comunidad de \acrshort{sfml}, accediendo a su foro, su wiki y a otros recursos encontrados en internet.

	Por otro lado, el alumno no tenía conocimiento alguno sobre la generación procedural. Sin embargo, era un área en la que estaba muy interesado, porque el hecho de ofrecer una experiencia única en cada partida lo seducía. Por ello, se lanzó a investigar sobre distintos algoritmos de generación, enfocándose principalmente en los de generación de mapas. En este punto huelga decir que el alumno se sorprendió con la cantidad de recursos que había al respecto, no se esperaba encontrar tantísima información. Tras barajar distintas posibilidades, se decantó por utilizar un algoritmo que, pese a su sencillez, otorga unos resultados muy satisfactorios. Esta decisión se tomó en base a que por desgracia no se disponía tanto tiempo como se hubiese deseado para el desarrollo.

	En general, hay que decir que pese a que se poseía conocimiento previo respecto a muchas áreas utilizadas como la programación, manipulación gráfica o el desarrollo de videojuegos, el alumno considera que ha sido una experiencia muy enriquecedora, y que gracias a ella ha aprendido un montón de técnicas y ha descubierto muchos recursos de aprendizaje nuevos que, aunque no ha podido utilizarlos todos en el desarrollo de este proyecto, sin duda serán de gran ayuda y de un valor incalculable para el futuro.

\section{Líneas Futuras}

	Aunque se han completado todos los objetivos definidos para el proyecto, hay ciertas extensiones que se podrían llevar a cabo para mejorar el producto. El juego se puede extender con múltiples características, algunas de ellas podrían ser:

	\begin{itemize}

		

	\end{itemize}
	