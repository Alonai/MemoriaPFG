\chapter{Introducción}\label{cha:introduccion}

\section{Presentación del Documento}

El presente informe describe el proyecto de desarrollo de Kingdom of Hatred, un videojuego en dos dimensiones con niveles generados proceduralmente detallando tanto los objetivos que se pretenden alcanzar con el proyecto, como las fases, actividades y recursos necesarios para llevarlo a cabo.

El contenido de este documento se estructura en torno a los siguientes apartados:

\begin{itemize}
	\item \textbf{Introducción:}
		
	Definición del contenido del documento, resumen del estado del arte y motivación.
	
	\item \textbf{Objetivos del proyecto:}
		
	Establecimiento de los objetivos del proyecto, su alcance y las tareas a realizar.
	
	\item \textbf{Especificación de requisitos:}

	Descripción de los requisitos del proyecto, analizados desde distintos puntos de vista.

	\item \textbf{Especificación del diseño:}
	
	Descripción de la solución elegida y cómo ha sido aplicada.

	\item \textbf{Consideraciones de la implementación:}

	Descripción de los aspectos más destacables de la implementación.

	\item \textbf{Plan de pruebas:}

	Definición del plan de pruebas utilizado para garantizar la calidad del producto.

	\item \textbf{Manual de usuario:}

	Descripción del uso del producto desarrollado al usuario.

	\item \textbf{Incidencias:}

	Descripción de los problemas encontrados en el desarrollo y cómo de han solucionado.

	\item \textbf{Conclusiones:}

	Análisis de los objetivos alcanzados y consideraciones adicionales.
\end{itemize}

\section{Estado del arte y motivación}

(Hablo de como esta la industria, de frameowrks, motores y por que he elegido hacerlo asi)

Este proyecto nace de la afición a los productos de entretenimiento digital y a la creación de los mismos, en concreto, al género de los juegos en dos dimensiones. El proyecto va a consistir en el desarrollo completo de un juego de este tipo, desde el añalisis de requisitos, diseño del juego y del software y su implementación. Además, los niveles del juegos tendrán que ser generados proceduralmente, así que se deberán implementar algoritmos adecuados para estos propósitos, junto con los demás requisitos típicos de un software tradicional (usabilidad, estabilidad...). El resultado principal consistirá en un juego de calidad, especialmente en el apartado de software, que pudiese competir con productos similares del mercado, además de servir como experiencia de aprendizaje para el desarrollo de futuros proyectos de esta índole.