\chapter*{Resumen}

Hoy en día, la industria del videojuego o entretenimiento digital está en pleno auge, y es una 	opción laboral más que válida. Sin embargo, al ser un área con tanta gente aficionada, 	existen muchos proyectos y empresas dedicadas a ello, y su número está aumentando. Por 	este motivo, la competitividad es muy grande para los nuevos proyectos que surjan, y es 	difícil hacerse un hueco en la industria. A pesar de que existen varias ofertas educativas para 	formarse en el tema y así introducirse en el mercado laboral, los expertos coinciden en que 	la manera más eficaz de dar el salto es creando juegos. Este proyecto servirá para crear un 	producto de entretenimiento, pero intentando crear una arquitectura de software lo más 	adecuada posible, y utilizando tecnologías propias de la mencionada industria. El resultado 	será un producto de carácter profesional que proporcionará experiencia para desarrollar 	futuros proyectos e incluso pudiera ser comercializable.

\vspace{2em}

{\Large\bfseries\sffamily Descriptores}
\vspace{3\medskipamount}

Videojuego, 2D, generación procedural.