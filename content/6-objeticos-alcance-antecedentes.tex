\chapter{Objetivos y alcance}

\section{Objetivos}

El objetivo principal de este proyecto es conseguir un producto jugable y estable. Por tanto, se deben llevar a cabo un análisis de requisitos, diseño e implementación. No necesariamente el juego debe estar terminado, pero es necesario que las características principales estén implementadas y pulidas. Por otro lado, la arquitectura del software debe estar preparada para ser fácilmente escalable y ampliable.

En cuanto a los objetivos secundarios del proyecto, el primero es puramente didáctico. Debido a la naturaleza de este tipo de software, el cual debe tener una respuesta en tiempo real, estable y funcionar en una gama grande de hardware. Por estos motivos, el software debe estar construido de manera específica, y el objetivo es aprender a crear arquitecturas aptas para este tipo de aplicaciones. En segundo lugar, se quieren aprender y aplicar técnicas y patrones de diseño software conveniente en este ámbito. Por último, se quiere realizar un producto de calidad que añadir al currículum.


\section{Alcance}

Atendiendo a las premisas señaladas anteriormente, las funcionalidades que deberá soportar Kingdom of Hatred serán:

\begin{itemize}
	\item Una arquitectura de software que permita añadir, eliminar y modificar elementos fácilmente.
	
	\item Niveles generados proceduralmente.

	\item Una interfaz gráfica de usuario que permita acceder a las partidas y mostrar controles.

	\item Una experiencia de juego pulida.
\end{itemize}

\chapter{Producto final}

El producto final será una demo técnica, en la cual deberán estar implementadas las 	funcionalidades principales del juego. Al iniciarse el juego se mostrará el menú principal, el 	cual contará con las siguientes opciones:

\begin{itemize}
	\item \textbf{Play:}

	Ejecutará el juego, el cual tendrá las siguientes características:

	\begin{itemize}
			\item El mapa del juego será generado aleatoriamente.
			\item El mapa estará poblado de enemigos, cuya localización también será aleatoria. Habrá distintos tipos de enemigos, y todos intentarán dañar al jugador.
			\item El jugador podrá moverse, realizar un ataque básico cuerpo a cuerpo y dispondrá de dos habilidades. Todos estos mecanismos serán los que utilizará para acabar con los enemigos.
			\item El juego finaliza cuando el jugador es abatido o cuando todos los monstruos son derrotados.
	\end{itemize}
	\item \textbf{Credits:}

	Mostrará una pantalla estática con la información referente a los desarrolladores y los orígenes de los gráficos y sonidos utilizados.

	\item \textbf{Exit:}

	Cerrará la aplicación.
\end{itemize}