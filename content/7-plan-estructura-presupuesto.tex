\chapter{Descripción de realización}

\section{Método de desarrollo}

Kingdom of Hatred se desarrollará mediante un sistema iterativo e incremental. Este proceso de desarrollo suple las carencias del modelo de cascada, el modelo tradicional que establece una rigorosa jerarquía en las fases del desarrollo y requiere completar una fase para comenzar la siguiente. En la Figura \ref{fig:modeloI} se puede observar el diagrama del modelo incremental.

\begin{figure}[!htp]
 \centering
 \includegraphics{fig/modelo_incremental}
 \caption{Modelo incremental}
 \label{fig:modeloI}
\end{figure}

El desarrollo incremental permite desarrollar una parte funcional del proyecto en cada etapa, reservando la mejora o extensión de funcionalidades para el futuro y por tanto controlando la complejidad y los riesgos. Además, este sistema permite a los desarrolladores aprovechar conocimiento adquirido en etapas previas e incorporar nuevo conocimiento y nuevas técnicas en fases venideras.

Adicionalmente, esta metodología confía en el desarrollo guiado por tests. Esta práctica consiste en el desarrollo de tests antes que código, y después se genera el mínimo código posible para completar esos tests. El objetivo de esta metodología es lograr código limpio y funcional, la idea es que los requisitos se traducen a evidencia, de forma que si los tests se completan satisfactoriamente, se garantiza que el software cubre dicho requisitos.

\subsection{Productos intermedios}

Los productos intermedios que se generarán en cada una de las fases son:

\begin{itemize}
	\item \textbf{Diseño del juego:}
	\begin{itemize}
		\item Documento de diseño de juego.
	\end{itemize}
	\item \textbf{Desarrollo del software:}
	\begin{itemize}
		\item Documento con la especificación del software.
	\end{itemize}
	\item \textbf{Validación técnica y usabilidad:}
	\begin{itemize}
		\item Informe de evaluación del juego.
	\end{itemize}
\end{itemize}

\subsection{EDT}

\begin{figure}[!htp]
	\centering
	\includegraphics[angle=90, scale=.5]{fig/edt}
	\caption{EDT}
\end{figure}

\section{Tareas principales}

El desarrollo de Kingdom of Hatred comprende las siguientes tareas o actividades:

\subsection{Lanzamiento del proyecto}

\begin{itemize}
	\item \textbf{Organización}

	Actividad mediante la que se define y prepara la planificación, asignación de misiones y el lanzamiento del proyecto y sus sucesivas fases.
	\item \textbf{Seguimiento}

	Realización del seguimiento y control del desarrollo del proyecto, que permita la rápida detención y solución de problemas que puedan dificultar la buena marcha del mismo.
\end{itemize}

\subsection{Análisis de herramientas y técnicas}

\begin{itemize}
	\item \textbf{Análisis de herramientas para desarrollo de juegos}
	
	Investigar distintas alternativas que existen para el desarrollo de juegos.

	\item \textbf{Análisis de técnicas de generación procedural}
	Investigar distintos algoritmos de generación procedural que sean adecuados para la generación de niveles.
\end{itemize}

\subsection{Diseño del juego}

\begin{itemize}
	\item Diseño del juego
	Crear el documento de desarrollo de juego que definirá cómo será este.
\end{itemize}

\subsection{Desarrollo del software}

\begin{itemize}
	\item \textbf{Formación}
	Aprendizaje en la creación de juegos.
	\item \textbf{Diseño}
	Diseño del software del juego.
	\item \textbf{Implementación}
	Implementación del juego.
\end{itemize}

\subsection{Validación técnica y de usabilidad:}

\begin{itemize}
	\item \textbf{Betatesting}
	Uso intensivo del juego en busca de bugs.
	\item \textbf{Prueba de experiencia de usuario}
	Recoger opiniones para mejorar la experiencia de usuario.
\end{itemize}

\subsection{Distribución y cierre del proyecto}

\begin{itemize}
	\item \textbf{Despliegue de la versión final}
	
	Preparar el juego para el usuario final.
	\item \textbf{Cierre del proyecto}

	Cierre del proyecto.
\end{itemize}

\section{Hoja de Tareas}

% Tarea 1 %

\begin{center}
	\captionof{table}{Tareas 1-10}
	\begin{tabularx}{\textwidth}{@{\extracolsep{\fill} } | X | X | X |}
	\noalign{\hrule height 4pt}
	\multicolumn{3}{!{\vrule width 4pt} c !{\vrule width 4pt}}{\textbf{HOJA DE TAREAS}}	\\ \noalign{\hrule height 2pt}
	\multicolumn{3}{!{\vrule width 4pt} l !{\vrule width 4pt}}{\textbf{Nombre:} Unai Alonso Alvarez}	\\
	\multicolumn{3}{!{\vrule width 4pt} l !{\vrule width 4pt}}{\textbf{Fecha:} Marzo de 2015}			\\	\noalign{\hrule height 2pt}
	\multicolumn{2}{!{\vrule width 4pt} l}
	{
		\textbf{Identificación de Tarea:} T1-10
	}	&
		\multicolumn{1}{!{\vrule width 2pt} l !{\vrule width 4pt}}
		{
			\textbf{Duración:} 1 día
		}	\\	\Cline{2pt}{3-3}
	\multicolumn{2}{!{\vrule width 4pt} l}{\textbf{Descripción:}}	&
		\multicolumn{1}{!{\vrule width 2pt} l !{\vrule width 4pt}}
		{
			\textbf{Esfuerzo:} 3 horas
		}	\\
	\multicolumn{2}{!{\vrule width 4pt} p{0.7\linewidth}}
	{
		Seguimiento
	} &
		\multicolumn{1}{!{\vrule width 2pt} l !{\vrule width 4pt}}{}	\\	\noalign{\hrule height 2pt} 
	\multicolumn{2}{!{\vrule width 4pt} l}{\textbf{Criterios de Terminación:}}	&
		\multicolumn{1}{!{\vrule width 2pt} l !{\vrule width 4pt}}{\textbf{Tareas previas:}}	\\
	\multicolumn{2}{!{\vrule width 4pt} p{0.7\linewidth}}
	{
		Cuando el jefe de proyecto concluya que las tareas marchan adecuadamente respecto al plan de trabajo, o en caso contrario, cuando se haya definido y validado un plan de accion por el mismo jefe de proyecto.
	}	&
		\multicolumn{1}{!{\vrule width 2pt} l !{\vrule width 4pt}}
		{
			T11
		}	\\	\noalign{\hrule height 2pt}
	\multicolumn{2}{!{\vrule width 4pt} l}{\textbf{Competencias, conocimientos y notas:}} &
		\multicolumn{1}{!{\vrule width 2pt} l !{\vrule width 4pt}}{\textbf{Recursos:}}	\\
	\multicolumn{2}{!{\vrule width 4pt} p{0.7\linewidth}}
	{
		Persona encargada de dirigir el proyecto y con conocimiento de la marcha del proyecto.
	}	&
		\multicolumn{1}{!{\vrule width 2pt} p{0.2\textwidth} !{\vrule width 4pt}}
		{
			Jefe de proyecto
			
			Programador
		}	\\
	\noalign{\hrule height 4pt}
	\end{tabularx}
\end{center}

\clearpage

% Tarea 2 %

\begin{center}
	\captionof{table}{Tarea 11}
	\begin{tabularx}{\textwidth}{@{\extracolsep{\fill} } | X | X | X |}
	\noalign{\hrule height 4pt}
	\multicolumn{3}{!{\vrule width 4pt} c !{\vrule width 4pt}}{\textbf{HOJA DE TAREAS}}	\\ \noalign{\hrule height 2pt}
	\multicolumn{3}{!{\vrule width 4pt} l !{\vrule width 4pt}}{\textbf{Nombre:} Unai Alonso Alvarez}	\\
	\multicolumn{3}{!{\vrule width 4pt} l !{\vrule width 4pt}}{\textbf{Fecha:} Marzo de 2015}			\\	\noalign{\hrule height 2pt}
	\multicolumn{2}{!{\vrule width 4pt} l}
	{
		\textbf{Identificación de Tarea:} T11
	}	&
		\multicolumn{1}{!{\vrule width 2pt} l !{\vrule width 4pt}}
		{
			\textbf{Duración:} 2 días
		}	\\	\Cline{2pt}{3-3}
	\multicolumn{2}{!{\vrule width 4pt} l}{\textbf{Descripción:}}	&
		\multicolumn{1}{!{\vrule width 2pt} l !{\vrule width 4pt}}
		{
			\textbf{Esfuerzo:} 6 horas
		}	\\
	\multicolumn{2}{!{\vrule width 4pt} p{0.7\linewidth}}
	{
		Organización
	} &
		\multicolumn{1}{!{\vrule width 2pt} l !{\vrule width 4pt}}{}	\\	\noalign{\hrule height 2pt} 
	\multicolumn{2}{!{\vrule width 4pt} l}{\textbf{Criterios de Terminación:}}	&
		\multicolumn{1}{!{\vrule width 2pt} l !{\vrule width 4pt}}{\textbf{Tareas previas:}}	\\
	\multicolumn{2}{!{\vrule width 4pt} p{0.7\linewidth}}
	{
		Se ha definido la estrategia inicial para el desarrollo del proyecto y ha sido validada por el director del proyecto.
	}	&
		\multicolumn{1}{!{\vrule width 2pt} l !{\vrule width 4pt}}
		{
			Ninguna
		}	\\	\noalign{\hrule height 2pt}
	\multicolumn{2}{!{\vrule width 4pt} l}{\textbf{Competencias, conocimientos y notas:}} &
		\multicolumn{1}{!{\vrule width 2pt} l !{\vrule width 4pt}}{\textbf{Recursos:}}	\\
	\multicolumn{2}{!{\vrule width 4pt} p{0.7\linewidth}}
	{
		Persona encargada de dirigir el proyecto.
	}	&
		\multicolumn{1}{!{\vrule width 2pt} p{0.2\textwidth} !{\vrule width 4pt}}
		{
			Jefe de proyecto
		}	\\
	\noalign{\hrule height 4pt}
	\end{tabularx}
\end{center}

% Tarea 3 %

\begin{center}
	\captionof{table}{Tarea 12}
	\begin{tabularx}{\textwidth}{@{\extracolsep{\fill} } | X | X | X |}
	\noalign{\hrule height 4pt}
	\multicolumn{3}{!{\vrule width 4pt} c !{\vrule width 4pt}}{\textbf{HOJA DE TAREAS}}	\\ \noalign{\hrule height 2pt}
	\multicolumn{3}{!{\vrule width 4pt} l !{\vrule width 4pt}}{\textbf{Nombre:} Unai Alonso Alvarez}	\\
	\multicolumn{3}{!{\vrule width 4pt} l !{\vrule width 4pt}}{\textbf{Fecha:} Marzo de 2015}			\\	\noalign{\hrule height 2pt}
	\multicolumn{2}{!{\vrule width 4pt} l}
	{
		\textbf{Identificación de Tarea:} T12
	}	&
		\multicolumn{1}{!{\vrule width 2pt} l !{\vrule width 4pt}}
		{
			\textbf{Duración:} 3 días
		}	\\	\Cline{2pt}{3-3}
	\multicolumn{2}{!{\vrule width 4pt} l}{\textbf{Descripción:}}	&
		\multicolumn{1}{!{\vrule width 2pt} l !{\vrule width 4pt}}
		{
			\textbf{Esfuerzo:} 9 horas
		}	\\
	\multicolumn{2}{!{\vrule width 4pt} p{0.7\linewidth}}
	{
		Análisis de herramientas de desarrollo de juegos
	} &
		\multicolumn{1}{!{\vrule width 2pt} l !{\vrule width 4pt}}{}	\\	\noalign{\hrule height 2pt} 
	\multicolumn{2}{!{\vrule width 4pt} l}{\textbf{Criterios de Terminación:}}	&
		\multicolumn{1}{!{\vrule width 2pt} l !{\vrule width 4pt}}{\textbf{Tareas previas:}}	\\
	\multicolumn{2}{!{\vrule width 4pt} p{0.7\linewidth}}
	{
		 El programador ha decidido cuáles serán las herramientas a utilizar para el desarrollo del software en base a sus características y el jefe de proyecto ha validado que son adecuadas.
	}	&
		\multicolumn{1}{!{\vrule width 2pt} l !{\vrule width 4pt}}{T11}	\\	\noalign{\hrule height 2pt}
	\multicolumn{2}{!{\vrule width 4pt} l}{\textbf{Competencias, conocimientos y notas:}} &
		\multicolumn{1}{!{\vrule width 2pt} l !{\vrule width 4pt}}{\textbf{Recursos:}}	\\
	\multicolumn{2}{!{\vrule width 4pt} p{0.7\linewidth}}
	{
		Persona que se encargará de desarrollar el software.
	}	&
		\multicolumn{1}{!{\vrule width 2pt} p{0.2\textwidth} !{\vrule width 4pt}}
		{
			Ordenador

			Navegador de Internet

			Programador

		}	\\
	\noalign{\hrule height 4pt}
	\end{tabularx}
\end{center}

\clearpage

% Tarea 4 %

\begin{center}
	\captionof{table}{Tarea 13}
	\begin{tabularx}{\textwidth}{@{\extracolsep{\fill} } | X | X | X |}
	\noalign{\hrule height 4pt}
	\multicolumn{3}{!{\vrule width 4pt} c !{\vrule width 4pt}}{\textbf{HOJA DE TAREAS}}	\\ \noalign{\hrule height 2pt}
	\multicolumn{3}{!{\vrule width 4pt} l !{\vrule width 4pt}}{\textbf{Nombre:} Unai Alonso Alvarez}	\\
	\multicolumn{3}{!{\vrule width 4pt} l !{\vrule width 4pt}}{\textbf{Fecha:} Marzo de 2015}			\\	\noalign{\hrule height 2pt}
	\multicolumn{2}{!{\vrule width 4pt} l}
	{
		\textbf{Identificación de Tarea:} T13
	}	&
		\multicolumn{1}{!{\vrule width 2pt} l !{\vrule width 4pt}}
		{
			\textbf{Duración:} 3 días
		}	\\	\Cline{2pt}{3-3}
	\multicolumn{2}{!{\vrule width 4pt} l}{\textbf{Descripción:}}	&
		\multicolumn{1}{!{\vrule width 2pt} l !{\vrule width 4pt}}
		{
			\textbf{Esfuerzo:} 9 horas
		}	\\
	\multicolumn{2}{!{\vrule width 4pt} p{0.7\linewidth}}
	{
		 Análisis de algoritmos de generación procedural
	} &
		\multicolumn{1}{!{\vrule width 2pt} l !{\vrule width 4pt}}{}	\\	\noalign{\hrule height 2pt} 
	\multicolumn{2}{!{\vrule width 4pt} l}{\textbf{Criterios de Terminación:}}	&
		\multicolumn{1}{!{\vrule width 2pt} l !{\vrule width 4pt}}{\textbf{Tareas previas:}}	\\
	\multicolumn{2}{!{\vrule width 4pt} p{0.7\linewidth}}
	{
		El programador ha decidido cuáles serán los algoritmos a utilizar para la generación aleatoria de escenarios y el jefe de proyecto ha validado que son adecuadas.
	}	&
		\multicolumn{1}{!{\vrule width 2pt} l !{\vrule width 4pt}}{T12}	\\	\noalign{\hrule height 2pt}
	\multicolumn{2}{!{\vrule width 4pt} l}{\textbf{Competencias, conocimientos y notas:}} &
		\multicolumn{1}{!{\vrule width 2pt} l !{\vrule width 4pt}}{\textbf{Recursos:}}	\\
	\multicolumn{2}{!{\vrule width 4pt} p{0.7\linewidth}}
	{
		Persona que se encargará de implementar los algoritmos.
	}	&
		\multicolumn{1}{!{\vrule width 2pt} p{0.2\textwidth} !{\vrule width 4pt}}
		{
			Ordenador

			Navegador de internet

			Programador

		}	\\
	\noalign{\hrule height 4pt}
	\end{tabularx}
\end{center}

% Tarea 5 %

\begin{center}
	\captionof{table}{Tarea 14}
	\begin{tabularx}{\textwidth}{@{\extracolsep{\fill} } | X | X | X |}
	\noalign{\hrule height 4pt}
	\multicolumn{3}{!{\vrule width 4pt} c !{\vrule width 4pt}}{\textbf{HOJA DE TAREAS}}	\\ \noalign{\hrule height 2pt}
	\multicolumn{3}{!{\vrule width 4pt} l !{\vrule width 4pt}}{\textbf{Nombre:} Unai Alonso Alvarez}	\\
	\multicolumn{3}{!{\vrule width 4pt} l !{\vrule width 4pt}}{\textbf{Fecha:} Marzo de 2015}			\\	\noalign{\hrule height 2pt}
	\multicolumn{2}{!{\vrule width 4pt} l}
	{
		\textbf{Identificación de Tarea:} T14
	}	&
		\multicolumn{1}{!{\vrule width 2pt} l !{\vrule width 4pt}}
		{
			\textbf{Duración:} 5 días
		}	\\	\Cline{2pt}{3-3}
	\multicolumn{2}{!{\vrule width 4pt} l}{\textbf{Descripción:}}	&
		\multicolumn{1}{!{\vrule width 2pt} l !{\vrule width 4pt}}
		{
			\textbf{Esfuerzo:} 15 horas
		}	\\
	\multicolumn{2}{!{\vrule width 4pt} p{0.7\linewidth}}
	{
		Diseño del juego
	} &
		\multicolumn{1}{!{\vrule width 2pt} l !{\vrule width 4pt}}{}	\\	\noalign{\hrule height 2pt} 
	\multicolumn{2}{!{\vrule width 4pt} l}{\textbf{Criterios de Terminación:}}	&
		\multicolumn{1}{!{\vrule width 2pt} l !{\vrule width 4pt}}{\textbf{Tareas previas:}}	\\
	\multicolumn{2}{!{\vrule width 4pt} p{0.7\linewidth}}
	{
		Se ha producido un documento de diseño de juego completo y el director del proyecto lo aprueba.
	}	&
		\multicolumn{1}{!{\vrule width 2pt} l !{\vrule width 4pt}}{T11}	\\	\noalign{\hrule height 2pt}
	\multicolumn{2}{!{\vrule width 4pt} l}{\textbf{Competencias, conocimientos y notas:}} &
		\multicolumn{1}{!{\vrule width 2pt} l !{\vrule width 4pt}}{\textbf{Recursos:}}	\\
	\multicolumn{2}{!{\vrule width 4pt} p{0.7\linewidth}}
	{
		Persona encargada de diseñar el juego y dirigir su creación.
	}	&
		\multicolumn{1}{!{\vrule width 2pt} p{0.2\textwidth} !{\vrule width 4pt}}
		{
			Ordenador

			Jefe de proyecto
		}	\\
	\noalign{\hrule height 4pt}
	\end{tabularx}
\end{center}

% Tarea 6 %

\begin{center}
	\captionof{table}{Tarea 15}
	\begin{tabularx}{\textwidth}{@{\extracolsep{\fill} } | X | X | X |}
	\noalign{\hrule height 4pt}
	\multicolumn{3}{!{\vrule width 4pt} c !{\vrule width 4pt}}{\textbf{HOJA DE TAREAS}}	\\ \noalign{\hrule height 2pt}
	\multicolumn{3}{!{\vrule width 4pt} l !{\vrule width 4pt}}{\textbf{Nombre:} Unai Alonso Alvarez}	\\
	\multicolumn{3}{!{\vrule width 4pt} l !{\vrule width 4pt}}{\textbf{Fecha:} Marzo de 2015}			\\	\noalign{\hrule height 2pt}
	\multicolumn{2}{!{\vrule width 4pt} l}
	{
		\textbf{Identificación de Tarea:} T15
	}	&
		\multicolumn{1}{!{\vrule width 2pt} l !{\vrule width 4pt}}
		{
			\textbf{Duración:} 15 días
		}	\\	\Cline{2pt}{3-3}
	\multicolumn{2}{!{\vrule width 4pt} l}{\textbf{Descripción:}}	&
		\multicolumn{1}{!{\vrule width 2pt} l !{\vrule width 4pt}}
		{
			\textbf{Esfuerzo:} 45 horas
		}	\\
	\multicolumn{2}{!{\vrule width 4pt} p{0.7\linewidth}}
	{
		Formación
	} &
		\multicolumn{1}{!{\vrule width 2pt} l !{\vrule width 4pt}}{}	\\	\noalign{\hrule height 2pt} 
	\multicolumn{2}{!{\vrule width 4pt} l}{\textbf{Criterios de Terminación:}}	&
		\multicolumn{1}{!{\vrule width 2pt} l !{\vrule width 4pt}}{\textbf{Tareas previas:}}	\\
	\multicolumn{2}{!{\vrule width 4pt} p{0.7\linewidth}}
	{
		El programador conoce y es capaz de desenvolverse con las herramientas a utilizar.
	}	&
		\multicolumn{1}{!{\vrule width 2pt} l !{\vrule width 4pt}}
		{
			T13
		}	\\	\noalign{\hrule height 2pt}
	\multicolumn{2}{!{\vrule width 4pt} l}{\textbf{Competencias, conocimientos y notas:}} &
		\multicolumn{1}{!{\vrule width 2pt} l !{\vrule width 4pt}}{\textbf{Recursos:}}	\\
	\multicolumn{2}{!{\vrule width 4pt} p{0.7\linewidth}}
	{
		Persona encargada de implementar el juego.
	}	&
		\multicolumn{1}{!{\vrule width 2pt} l !{\vrule width 4pt}}{\parbox{0.2\textwidth}
		{
			Ordenador

			Programador
		}}	\\
	\noalign{\hrule height 4pt}
	\end{tabularx}
\end{center}

% Tarea 7 %

\begin{center}
	\captionof{table}{Tarea 16}
	\begin{tabularx}{\textwidth}{@{\extracolsep{\fill} } | X | X | X |}
	\noalign{\hrule height 4pt}
	\multicolumn{3}{!{\vrule width 4pt} c !{\vrule width 4pt}}{\textbf{HOJA DE TAREAS}}	\\ \noalign{\hrule height 2pt}
	\multicolumn{3}{!{\vrule width 4pt} l !{\vrule width 4pt}}{\textbf{Nombre:} Unai Alonso Alvarez}	\\
	\multicolumn{3}{!{\vrule width 4pt} l !{\vrule width 4pt}}{\textbf{Fecha:} Marzo de 2015}			\\	\noalign{\hrule height 2pt}
	\multicolumn{2}{!{\vrule width 4pt} l}
	{
		\textbf{Identificación de Tarea:} T16
	}	&
		\multicolumn{1}{!{\vrule width 2pt} l !{\vrule width 4pt}}
		{
			\textbf{Duración:} 15 días
		}	\\	\Cline{2pt}{3-3}
	\multicolumn{2}{!{\vrule width 4pt} l}{\textbf{Descripción:}}	&
		\multicolumn{1}{!{\vrule width 2pt} l !{\vrule width 4pt}}
		{
			\textbf{Esfuerzo:} 45 horas
		}	\\
	\multicolumn{2}{!{\vrule width 4pt} p{0.7\linewidth}}
	{
		Diseño del software.
	} &
		\multicolumn{1}{!{\vrule width 2pt} l !{\vrule width 4pt}}{}	\\	\noalign{\hrule height 2pt} 
	\multicolumn{2}{!{\vrule width 4pt} l}{\textbf{Criterios de Terminación:}}	&
		\multicolumn{1}{!{\vrule width 2pt} l !{\vrule width 4pt}}{\textbf{Tareas previas:}}	\\
	\multicolumn{2}{!{\vrule width 4pt} p{0.7\linewidth}}
	{
		Se ha generado un documento detallando el diseño del software y el director de proyecto lo aprueba.
	}	&
		\multicolumn{1}{!{\vrule width 2pt} l !{\vrule width 4pt}}
		{
			T14
			T15
		}	\\	\noalign{\hrule height 2pt}
	\multicolumn{2}{!{\vrule width 4pt} l}{\textbf{Competencias, conocimientos y notas:}} &
		\multicolumn{1}{!{\vrule width 2pt} l !{\vrule width 4pt}}{\textbf{Recursos:}}	\\
	\multicolumn{2}{!{\vrule width 4pt} p{0.7\linewidth}}
	{
		Persona encargada de implementar el juego.

	}	&
		\multicolumn{1}{!{\vrule width 2pt} l !{\vrule width 4pt}}{\parbox{0.2\textwidth}
		{
			Programador

			Ordenador
		}}	\\
	\noalign{\hrule height 4pt}
	\end{tabularx}
\end{center}

% tarea 8 %

\begin{center}
	\captionof{table}{Tarea 17}
	\begin{tabularx}{\textwidth}{@{\extracolsep{\fill} } | X | X | X |}
	\noalign{\hrule height 4pt}
	\multicolumn{3}{!{\vrule width 4pt} c !{\vrule width 4pt}}{\textbf{HOJA DE TAREAS}}	\\ \noalign{\hrule height 2pt}
	\multicolumn{3}{!{\vrule width 4pt} l !{\vrule width 4pt}}{\textbf{Nombre:} Unai Alonso Alvarez}	\\
	\multicolumn{3}{!{\vrule width 4pt} l !{\vrule width 4pt}}{\textbf{Fecha:} Marzo de 2015}			\\	\noalign{\hrule height 2pt}
	\multicolumn{2}{!{\vrule width 4pt} l}
	{
		\textbf{Identificación de Tarea:} T17
	}	&
		\multicolumn{1}{!{\vrule width 2pt} l !{\vrule width 4pt}}
		{
			\textbf{Duración:} 30 días
		}	\\	\Cline{2pt}{3-3}
	\multicolumn{2}{!{\vrule width 4pt} l}{\textbf{Descripción:}}	&
		\multicolumn{1}{!{\vrule width 2pt} l !{\vrule width 4pt}}
		{
			\textbf{Esfuerzo:} 90 horas
		}	\\
	\multicolumn{2}{!{\vrule width 4pt} p{0.7\linewidth}}
	{
		Implementación
	} &
		\multicolumn{1}{!{\vrule width 2pt} l !{\vrule width 4pt}}{}	\\	\noalign{\hrule height 2pt} 
	\multicolumn{2}{!{\vrule width 4pt} l}{\textbf{Criterios de Terminación:}}	&
		\multicolumn{1}{!{\vrule width 2pt} l !{\vrule width 4pt}}{\textbf{Tareas previas:}}	\\
	\multicolumn{2}{!{\vrule width 4pt} p{0.7\linewidth}}
	{
		Se ha generado una versión de software con todas las funcionalidades implementadas y el director lo aprueba.
	}	&
		\multicolumn{1}{!{\vrule width 2pt} l !{\vrule width 4pt}}
		{
			T16
		}	\\	\noalign{\hrule height 2pt}
	\multicolumn{2}{!{\vrule width 4pt} l}{\textbf{Competencias, conocimientos y notas:}} &
		\multicolumn{1}{!{\vrule width 2pt} l !{\vrule width 4pt}}{\textbf{Recursos:}}	\\
	\multicolumn{2}{!{\vrule width 4pt} p{0.7\linewidth}}
	{
		Persona encargada de implementar el juego.
	}	&
		\multicolumn{1}{!{\vrule width 2pt} l !{\vrule width 4pt}}{\parbox{0.2\textwidth}
		{
			Programador

			Ordenador
		}}	\\
	\noalign{\hrule height 4pt}
	\end{tabularx}
\end{center}

% Tarea 9 %

\begin{center}
	\captionof{table}{Tarea 18}
	\begin{tabularx}{\textwidth}{@{\extracolsep{\fill} } | X | X | X |}
	\noalign{\hrule height 4pt}
	\multicolumn{3}{!{\vrule width 4pt} c !{\vrule width 4pt}}{\textbf{HOJA DE TAREAS}}	\\ \noalign{\hrule height 2pt}
	\multicolumn{3}{!{\vrule width 4pt} l !{\vrule width 4pt}}{\textbf{Nombre:} Unai Alonso Alvarez}	\\
	\multicolumn{3}{!{\vrule width 4pt} l !{\vrule width 4pt}}{\textbf{Fecha:} Marzo de 2015}			\\	\noalign{\hrule height 2pt}
	\multicolumn{2}{!{\vrule width 4pt} l}
	{
		\textbf{Identificación de Tarea:} T18
	}	&
		\multicolumn{1}{!{\vrule width 2pt} l !{\vrule width 4pt}}
		{
			\textbf{Duración:} 15 días
		}	\\	\Cline{2pt}{3-3}
	\multicolumn{2}{!{\vrule width 4pt} l}{\textbf{Descripción:}}	&
		\multicolumn{1}{!{\vrule width 2pt} l !{\vrule width 4pt}}
		{
			\textbf{Esfuerzo:} 45 horas
		}	\\
	\multicolumn{2}{!{\vrule width 4pt} p{0.7\linewidth}}
	{
		Betatesting
	} &
		\multicolumn{1}{!{\vrule width 2pt} l !{\vrule width 4pt}}{}	\\	\noalign{\hrule height 2pt} 
	\multicolumn{2}{!{\vrule width 4pt} l}{\textbf{Criterios de Terminación:}}	&
		\multicolumn{1}{!{\vrule width 2pt} l !{\vrule width 4pt}}{\textbf{Tareas previas:}}	\\
	\multicolumn{2}{!{\vrule width 4pt} p{0.7\linewidth}}
	{
		El jefe de proyecto acepta el estado actual del juego como válido o se termina el tiempo asignado a la tarea
	}	&
		\multicolumn{1}{!{\vrule width 2pt} l !{\vrule width 4pt}}
		{
			T17
		}	\\	\noalign{\hrule height 2pt}
	\multicolumn{2}{!{\vrule width 4pt} l}{\textbf{Competencias, conocimientos y notas:}} &
		\multicolumn{1}{!{\vrule width 2pt} l !{\vrule width 4pt}}{\textbf{Recursos:}}	\\
	\multicolumn{2}{!{\vrule width 4pt} p{0.7\linewidth}}
	{
		Persona encargada de implementar el juego, que corregira los errores que se encuentren.

		Persona encargada de probar intensivamente el software en busca de errores.
	}	&
		\multicolumn{1}{!{\vrule width 2pt} l !{\vrule width 4pt}}{\parbox{0.2\textwidth}
		{
			Programador

			Tester

			Ordenador

			Juego
		}}	\\
	\noalign{\hrule height 4pt}
	\end{tabularx}
\end{center}

\clearpage

% Tarea 10 %

\begin{center}
	\captionof{table}{Tarea 19}
	\begin{tabularx}{\textwidth}{@{\extracolsep{\fill} } | X | X | X |}
	\noalign{\hrule height 4pt}
	\multicolumn{3}{!{\vrule width 4pt} c !{\vrule width 4pt}}{\textbf{HOJA DE TAREAS}}	\\ \noalign{\hrule height 2pt}
	\multicolumn{3}{!{\vrule width 4pt} l !{\vrule width 4pt}}{\textbf{Nombre:} Unai Alonso Alvarez}	\\
	\multicolumn{3}{!{\vrule width 4pt} l !{\vrule width 4pt}}{\textbf{Fecha:} Marzo de 2015}			\\	\noalign{\hrule height 2pt}
	\multicolumn{2}{!{\vrule width 4pt} l}
	{
		\textbf{Identificación de Tarea:} T19
	}	&
		\multicolumn{1}{!{\vrule width 2pt} l !{\vrule width 4pt}}
		{
			\textbf{Duración:} 15 días
		}	\\	\Cline{2pt}{3-3}
	\multicolumn{2}{!{\vrule width 4pt} l}{\textbf{Descripción:}}	&
		\multicolumn{1}{!{\vrule width 2pt} l !{\vrule width 4pt}}
		{
			\textbf{Esfuerzo:} 45 horas
		}	\\
	\multicolumn{2}{!{\vrule width 4pt} p{0.7\linewidth}}
	{
		Prueba de experiencia de usuario
	} &
		\multicolumn{1}{!{\vrule width 2pt} l !{\vrule width 4pt}}{}	\\	\noalign{\hrule height 2pt} 
	\multicolumn{2}{!{\vrule width 4pt} l}{\textbf{Criterios de Terminación:}}	&
		\multicolumn{1}{!{\vrule width 2pt} l !{\vrule width 4pt}}{\textbf{Tareas previas:}}	\\
	\multicolumn{2}{!{\vrule width 4pt} p{0.7\linewidth}}
	{
		El jefe de proyecto acepta la experiencia actual como válida o se ha terminado el tiempo asignado para la tarea.
	}	&
		\multicolumn{1}{!{\vrule width 2pt} l !{\vrule width 4pt}}
		{
			T18
		}	\\	\noalign{\hrule height 2pt}
	\multicolumn{2}{!{\vrule width 4pt} l}{\textbf{Competencias, conocimientos y notas:}} &
		\multicolumn{1}{!{\vrule width 2pt} l !{\vrule width 4pt}}{\textbf{Recursos:}}	\\
	\multicolumn{2}{!{\vrule width 4pt} p{0.7\linewidth}}
	{
		Persona encargada de implementar el juego, que modificara lo que sea necesario.

		Personas encargadas de probar el software y definir su usabilidad.
	}	&
		\multicolumn{1}{!{\vrule width 2pt} l !{\vrule width 4pt}}{\parbox{0.2\textwidth}
		{
			Programador

			Tester

			Ordenador

			Juego
		}}	\\
	\noalign{\hrule height 4pt}
	\end{tabularx}
\end{center}

% Tarea 11 %

\begin{center}
	\captionof{table}{Tarea 20}
	\begin{tabularx}{\textwidth}{@{\extracolsep{\fill} } | X | X | X |}
	\noalign{\hrule height 4pt}
	\multicolumn{3}{!{\vrule width 4pt} c !{\vrule width 4pt}}{\textbf{HOJA DE TAREAS}}	\\ \noalign{\hrule height 2pt}
	\multicolumn{3}{!{\vrule width 4pt} l !{\vrule width 4pt}}{\textbf{Nombre:} Unai Alonso Alvarez}	\\
	\multicolumn{3}{!{\vrule width 4pt} l !{\vrule width 4pt}}{\textbf{Fecha:} Marzo de 2015}			\\	\noalign{\hrule height 2pt}
	\multicolumn{2}{!{\vrule width 4pt} l}
	{
		\textbf{Identificación de Tarea:} T20
	}	&
		\multicolumn{1}{!{\vrule width 2pt} l !{\vrule width 4pt}}
		{
			\textbf{Duración:} 5 días
		}	\\	\Cline{2pt}{3-3}
	\multicolumn{2}{!{\vrule width 4pt} l}{\textbf{Descripción:}}	&
		\multicolumn{1}{!{\vrule width 2pt} l !{\vrule width 4pt}}
		{
			\textbf{Esfuerzo:} 15 horas
		}	\\
	\multicolumn{2}{!{\vrule width 4pt} p{0.7\linewidth}}
	{
		Despliegue de la versión final
	} &
		\multicolumn{1}{!{\vrule width 2pt} l !{\vrule width 4pt}}{}	\\	\noalign{\hrule height 2pt} 
	\multicolumn{2}{!{\vrule width 4pt} l}{\textbf{Criterios de Terminación:}}	&
		\multicolumn{1}{!{\vrule width 2pt} l !{\vrule width 4pt}}{\textbf{Tareas previas:}}	\\
	\multicolumn{2}{!{\vrule width 4pt} p{0.7\linewidth}}
	{
		Se ha generado un ejecutable que funcionará en cualquier ordenador de las plataformas destinadas.
	}	&
		\multicolumn{1}{!{\vrule width 2pt} l !{\vrule width 4pt}}
		{
			T19
		}	\\	\noalign{\hrule height 2pt}
	\multicolumn{2}{!{\vrule width 4pt} l}{\textbf{Competencias, conocimientos y notas:}} &
		\multicolumn{1}{!{\vrule width 2pt} l !{\vrule width 4pt}}{\textbf{Recursos:}}	\\
	\multicolumn{2}{!{\vrule width 4pt} p{0.7\linewidth}}
	{
		Persona encargada de implementar el juego, que generará la versión final.
	}	&
		\multicolumn{1}{!{\vrule width 2pt} l !{\vrule width 4pt}}{\parbox{0.2\textwidth}
		{
			Programador

			Ordenador

			Juego
		}}	\\
	\noalign{\hrule height 4pt}
	\end{tabularx}
\end{center}

% Tarea 12 %

\begin{center}
	\captionof{table}{Tarea 21}
	\begin{tabularx}{\textwidth}{@{\extracolsep{\fill} } | X | X | X |}
	\noalign{\hrule height 4pt}
	\multicolumn{3}{!{\vrule width 4pt} c !{\vrule width 4pt}}{\textbf{HOJA DE TAREAS}}	\\ \noalign{\hrule height 2pt}
	\multicolumn{3}{!{\vrule width 4pt} l !{\vrule width 4pt}}{\textbf{Nombre:} Unai Alonso Alvarez}	\\
	\multicolumn{3}{!{\vrule width 4pt} l !{\vrule width 4pt}}{\textbf{Fecha:} Marzo de 2015}			\\	\noalign{\hrule height 2pt}
	\multicolumn{2}{!{\vrule width 4pt} l}
	{
		\textbf{Identificación de Tarea:} T21
	}	&
		\multicolumn{1}{!{\vrule width 2pt} l !{\vrule width 4pt}}
		{
			\textbf{Duración:} 1 día
		}	\\	\Cline{2pt}{3-3}
	\multicolumn{2}{!{\vrule width 4pt} l}{\textbf{Descripción:}}	&
		\multicolumn{1}{!{\vrule width 2pt} l !{\vrule width 4pt}}
		{
			\textbf{Esfuerzo:} 3 horas
		}	\\
	\multicolumn{2}{!{\vrule width 4pt} p{0.7\linewidth}}
	{
		Cierre del proyecto
	} &
		\multicolumn{1}{!{\vrule width 2pt} l !{\vrule width 4pt}}{}	\\	\noalign{\hrule height 2pt} 
	\multicolumn{2}{!{\vrule width 4pt} l}{\textbf{Criterios de Terminación:}}	&
		\multicolumn{1}{!{\vrule width 2pt} l !{\vrule width 4pt}}{\textbf{Tareas previas:}}	\\
	\multicolumn{2}{!{\vrule width 4pt} p{0.7\linewidth}}
	{
		Todas las tareas del proyecto han sido terminadas y se ha aprendido de la experiencia.
	}	&
		\multicolumn{1}{!{\vrule width 2pt} l !{\vrule width 4pt}}
		{
			T10, T20
		}	\\	\noalign{\hrule height 2pt}
	\multicolumn{2}{!{\vrule width 4pt} l}{\textbf{Competencias, conocimientos y notas:}} &
		\multicolumn{1}{!{\vrule width 2pt} l !{\vrule width 4pt}}{\textbf{Recursos:}}	\\
	\multicolumn{2}{!{\vrule width 4pt} p{0.7\linewidth}}
	{
		Programador, que aportará su experiencia.

		Jefe de proyecto, que se encargará de la gestión del conocimiento.
	}	&
		\multicolumn{1}{!{\vrule width 2pt} l !{\vrule width 4pt}}{\parbox{0.2\textwidth}
		{
			Programador

			Ordenador

			Jefe de proyecto
		}}	\\
	\noalign{\hrule height 4pt}
	\end{tabularx}
\end{center}

\chapter{Organización, Equipo}

\section{Esquema organizativo}

La organización del proyecto se articula en torno al comité dirección y al equipo de trabajo que se va a encargar de desarrollar el producto, en función de la estructura de la figura 5.1.

\begin{figure}[!htp]
	\centering
	\includegraphics[scale=.75]{fig/organization}
	\caption{Esquema organizativo}
\end{figure}

\begin{itemize}
	\item Comité de dirección: su función principal es orientar por dónde debería ir el proyecto y tomar las decisiones finales a la hora de qué hacer o no.  Además, este comité deberá aprobar las diferentes fases del proyecto.
	\item Equipo de trabajo: el órgano encargado de diseñar y desarrollar el contenido del proyecto en función de las diferentes fases estipuladas.
\end{itemize}

\section{Plan de Recursos Humanos}
\label{sec:planRecursosHumanos}

El equipo de trabajo estará formado por los siguientes perfiles directamente relacionados con las diferentes áreas de competencias que se abordan en el proyecto: 

\begin{itemize}
	\item Jefe de proyecto: su función es realizar las actividades de organización, coordinación y seguimiento del proyecto.
	\item Administrador de base de datos: su función es la de gestionar de una manera óptima la base de datos PostgreSQL y SPARQL y su función geoespacial. 
	\item Programador: su función es la desarrollar toda la lógica del programa como la implementación de la plataforma web. 
	\item Diseñador: su función es la diseñar interfaces intuitivas para el usuario y adaptables para distintos dispositivos (portátiles, tablets, móviles) 
	\item Experto en web semántica: su función es la de ayudar al equipo de trabajo a la hora de crear el sistema de búsquedas semánticas y facetadas. 
\end{itemize}

Debido a el bajo número de personas que compone el equipo de desarrollo se ha acordado trabajar mediante reuniones de seguimiento semanales pero también tras terminar cada tarea. En las reuniones semanales se reunirán todos los miembros del equipo, mientras que en las que corresponden a una tarea finalizada lo harán solo los que han participado en dicha tarea junto a el director de proyecto. Su finalidad será comentar los avances y/o problemas que hayan podido ocurrir, aunque también servirán para que el director de el visto bueno a la tarea y pasar a la siguiente. 

\chapter{Condiciones de ejecución}

\section{Entorno de trabajo}

El lugar de trabajo habitual serán las instalaciones de DeustoTech, aunque también se trabajará en casa para poder terminar a tiempo el proyecto.

El calendario y horario serán los correspondientes a los lugares de trabajo anteriormente mencionados durante una jornada laboral de aproximadamente 4 horas al día. Este horario podría verse modificado si se requiriera con el fin de cumplir los plazos establecidos.

En principio el director de proyecto será el responsable de todos los productos del desarrollo, y deberá dar el visto bueno a las herramientas que serán utilizadas para preservar las copias de seguridad y de definir cada cuanto tiempo deberán hacerse. En caso de que los desarrolladores no cumplan con estos requisitos y de producirse una perdida en el desarrollo serán estos los que asuman la responsabilidad, teniendo que optar por realizar horas extra o asumir de su sueldo la penalización que llegase a imponer el cliente en caso de no poder cumplirse con los plazos.

Los medios informáticos para la ejecución del proyecto deberán ser provistos por DeustoTech o serán los ordenadores personales de los integrantes del equipo. DeustoTech será responsable de todos los productos provistos para el desarrollo, salvo de aquellos medios pertenecientes a los propios desarrolladores. Los medios son los siguientes: 

\begin{itemize}
	\item Hardware
	\begin{itemize}
		\item Macbook Pro Retina 2012
		\item Servidor del repositorio Linux
		\item Monitor secundario
	\end{itemize}
	\item Software
	\begin{itemize}
		\item Licencia Sublime Text 2
		\item OS X
		\item Office 2011
		\item PostgreSQL
		\item SPARQL
	\end{itemize}
\end{itemize}

\section{Control de cambios}

Todas las peticiones que impliquen cambios en el diseño o en lo que ya está desarrollado, serán estudiadas y solo seguirán adelante si son modificaciones razonables y que son posibles de hacer dentro del plazo acordado. El procedimiento que habrá que seguir a la hora de  solicitar un cambio será:

\begin{enumerate}
	\item Comunicación de DeustoTech de las modificaciones solicitadas.
	\item Valoración por el equipo del proyecto de la repercusión técnica y cambios de plazos.
	\item Presentación de la decisión tomada por el equipo a DeustoTech.
	\item Notificación por parte de DeustoTech de la aprobación o no de la propuesta.	
	\item En caso afirmativo, modificación del plan de trabajo y del presupuesto.
\end{enumerate}

\section{Recepción de productos}

Para la recepción de productos el equipo del proyecto definirá una serie de pruebas que serán estrictamente ejecutadas. Una vez pasadas las pruebas, el jefe de proyecto deberá revisar y aceptar el producto para poder presentarlo oficialmente a DeustoTech.  En caso de que exista algún problema tras la revisión,  la dirección de DeustoTech-Internet deberá comunicarlo en un plazo máximo de 5 días para poder llevar a cabo las modificaciones y así poder seguir con la siguiente fase del proyecto. En caso de no obtener respuesta en el intervalo de tiempo especificado anteriormente, se considerará aprobado.

DeustoTech-Internet es el equipo de investigación centrado en el desarrollo web de la Universidad de Deusto. Este proyecto se ha delegado a varios de sus colaboradores de investigación. Dado a la estrecha relación que existen entre ambos no se han definido todos los requisitos desde el punto de partida, lo cual puede causar que retrasos en la fecha de entrega del producto. Sin embargo, al un proyecto interno no se le ha dado mayor importancia.

\chapter{Planificación}

\section{Diagrama de precedencias}

\begin{figure}[!htbp]
	\centering
	\includegraphics[page=1, scale=.65]{fig/real_network_diagram}
	\caption{Diagrama de precedencias 1}
\end{figure}

\begin{figure}[!htbp]
	\centering
	\includegraphics[page=2, scale=.65]{fig/real_network_diagram}
	\caption{Diagrama de precedencias 2}
\end{figure}

\begin{figure}[!htbp]
	\centering
	\includegraphics[page=3, scale=.65]{fig/real_network_diagram}
	\caption{Diagrama de precedencias 3}
\end{figure}

\begin{figure}[!htbp]
	\centering
	\includegraphics[page=4, scale=.65]{fig/real_network_diagram}
	\caption{Leyenda del diagrama de precedencias}
\end{figure}

\FloatBarrier

\section{Equipo Real}

El equipo real está compuesto por un único desarrollador, este desarrollador realizaría todas las tareas en cada uno de los roles descritos en la sección \ref{sec:planRecursosHumanos}.

\begin{center}
	\captionof{table}{Carga de trabajo del equipo real}
	\begin{tabular}{|l|l|l|r|}
		\hline
		Nombre & Inicio & Fin & Trabajo(h) \\ \hline
		Desarrollador & 02/03/2015 & 16/06/2015 & 308 \\
		\hline
	\end{tabular}
\end{center}

\clearpage

\section{Plan de trabajo}

\begin{figure}[!htp]
	\centering
	\includegraphics[page=1, scale=.8]{fig/real_work_plan_diagram}
	\caption{Diagrama del plan de trabajo 1}
\end{figure}

\begin{figure}[!htp]
	\centering
	\includegraphics[page=2, scale=.8]{fig/real_work_plan_diagram}
	\caption{Diagrama del plan de trabajo 2}
\end{figure}

\begin{figure}[!htp]
	\centering
	\includegraphics[page=3, scale=.8]{fig/real_work_plan_diagram}
	\caption{Diagrama del plan de trabajo 3}
\end{figure}

\begin{figure}[!htp]
	\centering
	\includegraphics[page=4, scale=.8]{fig/real_work_plan_diagram}
	\caption{Diagrama del plan de trabajo 4}
\end{figure}

\begin{figure}[!htp]
	\centering
	\includegraphics[page=5, scale=.8]{fig/real_work_plan_diagram}
	\caption{Diagrama del plan de trabajo 5}
\end{figure}

\begin{figure}[!htp]
	\centering
	\includegraphics[page=6, scale=.8]{fig/real_work_plan_diagram}
	\caption{Diagrama del plan de trabajo 6}
\end{figure}

\begin{figure}[!htp]
	\centering
	\includegraphics[page=7, scale=.8]{fig/real_work_plan_diagram}
	\caption{Diagrama del plan de trabajo 7}
\end{figure}

\begin{figure}[!htp]
	\centering
	\includegraphics[page=8, scale=.8]{fig/real_work_plan_diagram}
	\caption{Diagrama del plan de trabajo 8}
\end{figure}

\FloatBarrier

\section{Diagrama de Gantt}

\begin{figure}[!htp]
	\centering
	\includegraphics[page=1, scale=.7]{fig/real_gantt_diagram}
	\caption{Diagrama de Gantt 1}
\end{figure}

\begin{figure}[!htp]
	\centering
	\includegraphics[page=2, scale=.7]{fig/real_gantt_diagram}
	\caption{Diagrama de Gantt 2}
\end{figure}

\begin{figure}[!htp]
	\centering
	\includegraphics[page=3, scale=.7]{fig/real_gantt_diagram}
	\caption{Diagrama de Gantt 3}
\end{figure}

\begin{figure}[!htp]
	\centering
	\includegraphics[page=4, scale=.7]{fig/real_gantt_diagram}
	\caption{Leyenda del diagrama de Gantt}
\end{figure}

\FloatBarrier

\section{Estimación de cargas de trabajo por perfil}

\begin{center}
	\captionof{table}{Presupuesto: Cargas de trabajo por perfil}
	\begin{tabular}{|l|r|}
		\hline
		Perfil de trabajo & Carga de trabajo(h) \\ \hline
		Jefe de proyecto & 6,05 \\ \hline
		Administrador de base datos & 64,05 \\ \hline
		Diseñador gráfico & 26,22 \\ \hline
		Experto en web semántica & 26,22 \\ \hline
		Programador & 185,48 \\
		\hline
	\end{tabular}
\end{center}

\chapter{Presupuesto}

\section{Recursos Humanos}

\begin{center}
	\captionof{table}{Presupuesto: Recursos Humanos}
	\begin{tabular}{| l | r | r | r |}
		\hline
		Rol					&	Precio/hora(\euro/h)	&	Carga de trabajo(h)	&	Importe total(\euro)	\\	\hline
		Jefe de proyecto	& 	40						&	6,05 					& 	242,00				\\	\hline
		Administrador de base de datos &	25			&	64,05					&	1.601,25			\\	\hline
		Programador			&	25						&	185,48					&	4.637,00			\\	\hline
		Diseñador			&	15						&	26,22					&	393,30				\\	\hline
		Experto en web semántica	&		30			&	26,22					&	786,60				\\
		\hline
	\end{tabular}
\end{center}

\section{Recursos Software}

\begin{center}
	\captionof{table}{Presupuesto: Software}
	\begin{tabular}{| l | r | r | r |}
		\hline
		Nombre					&	Precio(\euro)	&	Unidades	&	Importe total(\euro)	\\	\hline
		Licencia Sublime Text 2	& 	70				&	1 			& 	70						\\	\hline
		Office 2011 			&	99				&	1			&	99						\\
		\hline
	\end{tabular}
\end{center}

\section{Recursos Hardware}

\begin{center}
	\captionof{table}{Presupuesto: Hardware}
	\begin{tabular}{| l | r | r | r |}
		\hline
		Nombre				&	Precio(\euro)	&	Unidades	&	Importe total(\euro)	\\	\hline
		MBPR2012			& 	3.334			&	1 			&	3.334					\\	\hline
		Monitor secundario	&	300				&	1			&	300						\\
		\hline
	\end{tabular}
\end{center}

\section{Total}

\begin{center}
	\captionof{table}{Presupuesto: Total}
	\begin{tabular}{| l | r |}
		\hline
		Tipo				&	Total			\\	\hline
		Recursos Humanos	& 	7.660,15		\\	\hline
		Recursos Software	&	169,00			\\	\hline
		Recursos Hardware	&	3.634,00		\\	\hline
		\textbf{Total}		&	\textbf{11.463,15}	\\
		\hline
	\end{tabular}
\end{center}