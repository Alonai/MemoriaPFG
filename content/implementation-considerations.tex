\chapter{Consideraciones sobre la implementación}

\section{Visión general}

\section{Reglas de estilo}

\section{Entorno de desarrollo}

	Para el desarrollo del proyecto se han utilizado varias herramientas para facilitar la codificación, compilación y búsqueda de errores.

	\subsection{Atom}

		Atom es un editor de texto desarrollado por Github. Es una herramienta que permite personalizar cualquier cosa, pero también permite ser usada productivamente sin tocar ningún archivo de configuración. Atom ofrece integración con Node.js y está diseñado de forma completamente modular, de forma que es posible acoplar un número indefinido de módulos a su núcleo mínimo. Además, su versión más mínima ya dispone de todas las características necesarias para empezar a trabajar. Por último, Atom es de código libre y, por tanto, gratuito.

		\begin{figure}[!htp]
			 \centering
			 \includegraphics{fig/atom}
			 \caption{Logo de Atom}
			 \label{fig:atom}
		\end{figure}

		Este software ha sido la herramienta principal para el proyecto. Ha sido utilizada para la codificación de todo el proyecto, para el retoque de los archivos JSON y para la generación de los archivos de configuración necesarios. Cualquier editor de texto sencillo podría haber sido utilizado, pero se ha optado por Atom ya que provee características avanzadas para codificar, tales como soporte para distintos lenguajes, que además pueden ser ampliadas mediante módulos gratuitos.

	\subsection{G++}

	\subsection{Git}

		Git es un software de control de versiones diseñado por Linus Torvalds pensando en la eficiencia y la confiabilidad del mantenimiento de versiones de aplicaciones cuando

	\subsection{Make}

	\subsection{Darkfunct Editor}

	\subsection{XML to JSON}

\section{Código y jerarquía de proyecto}

\section{Desarrollo del juego}