\chapter*{Resumen}

Deustotech cree que es posible contribuir a un mundo mejor mediante el uso de
las tecnologías de Internet y las Telecomunicaciones.

Como resultado de este pensamiento nació \acrshort{labman}, un sistema de gestión de grupo de investigación. Esta aplicación web tiene como objetivo gestionar toda la información referente a los investigadores, proyectos, publicaciones y tesis de un grupo relacionada entre si. Permite 
generar diversas gráficas que permiten analizar de forma rápida la evolución y desempeño del equipo
de investigación.
Este aplicativo en un claro ejemplo de una web de datos de nueva generación de portales web, dónde
no solo se exportan documentos, sino que habilita la exportación de datos y \acrshortpl{api}, que
tienen como propósito facilitar la explotación de recursos.

Aunque el sistema es capaz de exportar esta información semántica, todavía se ve la necesidad de que este colabore con sistemas imperantes en la industria para el intercambio de información de recursos académicos y científicos.
Es por ello que se requiere dar soporte a \acrshort{oai}, mediante la implementación de su protocolo \acrshort{oaipmh}, con el fin de dar servicio a las soluciones del sector que apostaron en su día por esta tecnología. Así mismo, se desea desarrollar un estudio sobre ventajas e inconvenientes que supone cada una de las tecnologías que usa LabMan para la explotación de la información, siendo estas \acrshort{sql} y \acrshort{sparql} en la actualidad y \acrshort{oaipmh} tras el despliegue en producción de este proyecto. Para finalizar este estudio, se dispondrá una conclusión general y la justificación de la tecnología que \acrshort{labman} usa como proveedor de servicios.

Por otra parte, se dispone de una aplicación web que tienen como objetivo principal la expansión
del sistema \acrshort{dms} actual en MORELab, permitiendo a los usuarios realizar búsquedas avanzadas de los recursos dispuestos por el servidor.

\vspace{2em}

{\Large\bfseries\sffamily Descriptores}
\vspace{3\medskipamount}

Biblioteca digital, Buscador avanzado, \acrshort{labman}, Aplicación Web, \acrshort{oaipmh}.