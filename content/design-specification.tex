\chapter{Especificación del Diseño}

\section{Visión general}

	En este capítulo se describe el trabajo de diseño realizado para el proyecto, así como el entorno y las herramientas que han sido utilizadas para llevarlo a cabo.

\section{Arquitectura}

	La arquitectura básica del juego es la que se muestra en la siguiente imagen. El usuario genera entradas al juego mediante el teclado, después estas entradas son procesadas por la lógica del juego y se realizan las acciones pertinentes en función del estado del juego. Una vez actualizado el estado del juego ha sido actualizado, los módulos de audio y vídeo generan las salidas correspondientes y las muestran mediante la pantalla y el dispositivo reproductor de audio actual.

	\begin{figure}[!htp]
		 \centering
		 \includegraphics{fig/architecture}
		 \caption{Diagrama de arquitectura}
		 \label{fig:arch}
	\end{figure}

	\FloatBarrier

\section{Diagramas de actividad}

	A continuación se muestra el diagrama de actividad del programa desde que se ejecuta hasta que finaliza.

	\begin{figure}[!htp]
		 \centering
		 \includegraphics{fig/actividad}
		 \caption{Diagrama de actividad}
		 \label{fig:activi}
	\end{figure}

	\FloatBarrier

\section{Diagrama de clases}
	
	//WIP

\section{Tecnologías utilizadas}

	En este capítulo se describirán las tecnologías utilizadas para el desempeño del proyecto. Además, se dará una breve explicación de por qué han sido utilizadas y qué beneficios aportan al proyecto.

	\subsection{\acrshort{sfml}}

		\acrfull{sfml} es una librería multiplataforma de desarrollo de software diseñada para proveer una interfaz simple a varios componentes multimedia en ordenadores. Está escrita en C++ con enlaces disponibles para C, D, Java, Python, Ruby, .NET, Go, Rust, OCaml, Euphoria y Nim. Existen también compilaciones experimentales para dispositivos móviles.

		\acrshort{sfml} gestiona tanto la creación e interacción de ventanas como de contextos \acrshort{opengl}. También provee de un módulo grñafico para gráficos acelerados por hardware en 2D el cual incluye representación de texto utilizando FreeType, un módulo de audio que se sirve de \acrshort{openal} y un módulo de conexión para comunicación básica por \acrshort{tcp} y \acrshort{udp}.

		\acrshort{sfml} es un software gratuito y de código libre provisto bajo los términos de la licencia zlib/png. Está disponible para Windows, Linux, OS X y FreeBSD.

		\begin{figure}[!htp]
			 \centering
			 \includegraphics[scale=.8]{fig/sfml}
			 \caption{Logotipo de \acrshort{sfml}}
			 \label{fig:sfml}
		\end{figure}

		\FloatBarrier

		Se ha decidido utilizar \acrshort{sfml} en el proyecto ya que uno de los objetivos del mismo es aprender a crear una arquitectura software apropiada para videojuegos, de forma que los motores con editor visual no eran una opción, al abstraer al usuario de ella. A pesar de que la librería de referencia para este tipo de aplicaciones es \acrshort{sdl}, se ha decidido usar \acrshort{sfml} ya que a diferencia de la primera, la cual está escrita en C, \acrshort{sfml} está escrita en C++ y concebida con orientación a objetos. Esto supone una ventaja ya que el proyecto ha sido desarrollado en C++ y el paradigma de programación utilizado ha sido el de la orientación a objetos.

	\subsection{C++}

		C++ es un lenguaje de programación de propósito general. Tiene características de programación imperativa, orientada a objetos y genérica, mientras provee facilidades para la manipulación de memoria a bajo nivel.

		Está diseñado pensando en la programación de sistemas, sistemas embebidos, sistemas con recursos limitados y grandes sistemas, con el rendimiento, la eficiencia y la flexibilidad de uso como sus requisitos de diseño. C++ también ha sido útil en otros muchos contextos, siendo fortalezas clave la infraestructura de software y aplicaciones con recursos limitados, incluyendo aplicaciones de escritorio, servidores, aplicaciones de rendimiento crítico y software de entretenimiento. C++ es un lenguaje compilado, con implementaciones del mismo disponibles en muchas plataformas y provistas por varias organizaciones, incluyendo \acrshort{fsf}, \acrshort{llvm}, Microsoft e Intel.

		C++ está estandarizado por \acrshort{iso}, con la última versión estándar ratificada y publicada por \acrshort{sfml} en diciembre de 2014 como \acrshort{iso}/\acrshort{iec} 14882:2014 (informalmente conocida como C++14). Muchos otros lenguajes de programación han sido influenciados por C++, entre los que se encuentran C\#, Java, y versiones posteriores a 1998 de C.

		\begin{figure}[!htp]
			 \centering
			 \includegraphics[scale=.25]{fig/cpp}
			 \caption{Logotipo de C++}
			 \label{fig:cpp}
		\end{figure}

		\FloatBarrier

		Se ha decidido utilizar C++ para el desarrollo del proyecto ya que es el lenguaje estándar de la industria, y dado que el producto busca competir, es importante que esté hecho con las mejores herramientas. C++ es en este caso dicha herramienta, por su eficiencia y flexibilidad.

	\subsection{\acrshort{json}}

		\acrfull{json} es un formato estándar abierto que usa texto legible por humanos para transmitir objetos de información consistentes en pares atributo-valor. Es principalmente utilizado para transmitir información entre un servidor y una aplicación web como alternativa a \acrshort{xml}.

		Aunque originalmente fue derivado del lenguaje de programación Javascript, \acrshort{json} es un formato de datos independiente del lenguaje. El código necesario para generar y analizar información en \acrshort{json} está disponible en muchos lenguajes de programación.

		\begin{figure}[!htp]
			 \centering
			 \includegraphics[scale=.5]{fig/json}
			 \caption{Logotipo de \acrshort{json}}
			 \label{fig:json}
		\end{figure}

		\FloatBarrier

		Se ha decidido utilizar \acrshort{json} frente a \acrshort{xml} porque el equipo disponía de experiencia previa con \acrshort{json} y no con \acrshort{xml}. Ambas tecnologías ofrecen características similares, pero \acrshort{json} es más simple y por tanto más fácil de mantener.

	\subsection{\acrshort{tgui}}

		\acrfull{tgui} es una librería de GUI multiplataforma en C++ para \acrshort{sfml}. Entre sus características más destacables encontramos la facilidad de uso, la portabilidad de código, la posibilidad de modificar interfaces sin necesidad de recompilar, un gestor de texturas externo que evita la recarga de imágenes y está provisto bajo la misma licencia que \acrshort{sfml}, zlib/png.

		\begin{figure}[!htp]
			 \centering
			 \includegraphics[scale=.75]{fig/tgui}
			 \caption{Logotipo de \acrshort{tgui}}
			 \label{fig:tgui}
		\end{figure}

		\FloatBarrier

		Se ha decidido utilizar \acrshort{tgui} en el proyecto para agilizar el desarrollo de las interfaces de usuario. Hubiese sido posible crear una implementación propia con los elementos necesarios, pero hubiese consumido recursos que eran de gran valor para otros apartados del proyecto.

	\subsection{JsonCpp}

		JsonCpp es una librería C++ que permite manipular documentos \acrshort{json}, incluyendo serialización y deserialización a y desde cadenas de texto. También preserva comentarios existentes en los pasos de serialización y deserialización, haciendo el formato conveniente para archivos generados por usuarios.

		De entre las distintas opciones para trabajar con \acrshort{json} en C++, se ha decantado por esta ya que su integración en el proyecto era muy sencilla, al igual que su uso.
