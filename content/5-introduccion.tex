\chapter{Introducción}

\section{Presentación del Documento}

El presente informe define el proyecto de desarrollo de Kingdom of Hatred, un videojuego en dos dimensiones con niveles generados proceduralmente detallando tanto los objetivos que se pretenden alcanzar con el proyecto, como las fases, actividades y recursos necesarios para llevarlo a cabo. 

El contenido de este documento se estructura en torno a los siguientes apartados: 

\begin{itemize}
	\item \textbf{Definición de proyecto:}
		
	Establecimiento del objetivo fundamental del proyecto, especificando cuáles son los aspectos funcionales que lo comprenden y cuáles son los que quedan excluidos.
	
	\item \textbf{Producto final:}
		
	Especificación de la solución elegida que va a construir el proyecto en cuestión.
	
	\item \textbf{Descripción de la realización:}

	Realización y definición de las diferentes actividades cuyo desarrollo va a permitir la realización y consecución del objetivo del proyecto.

	\item \textbf{Organización:}
	
	Definición del equipo de trabajo que desarrollará el proyecto, así como su estructura organizativa, sistema de gestión y seguimiento del trabajo.

	\item \textbf{Condiciones de ejecución:}

	Definición del entorno de trabajo, de los criterios sobre los que se van a realizar las sucesivas recepciones, así como el tratamiento que se va a establecer para aquellos casos que puedan ser considerados como modificaciones o mejoras en el planteamiento inicial del proyecto.

	\item \textbf{Planificación:}

	Estimación de cargas y duración de las diferentes actividades del proyecto, así como su asignación a los diferentes miembros del equipo y su planificación en el tiempo.

	\item \textbf{Valoración económica:}

	Determinación del valor correspondiente a este proyecto, de los hitos de facturación y de la forma de pago.
\end{itemize}

\section{Motivación}

Este proyecto nace de la afición a los productos de entretenimiento digital y a la creación de los mismos, en concreto, al género de los juegos en dos dimensiones. El proyecto va a consistir en el desarrollo completo de un juego de este tipo, desde el añalisis de requisitos, diseño del juego y del software y su implementación. Además, los niveles del juegos tendrán que ser generados proceduralmente, así que se deberán implementar algoritmos adecuados para estos propósitos, junto con los demás requisitos típicos de un software tradicional (usabilidad, estabilidad...). El resultado principal consistirá en un juego de calidad, especialmente en el apartado de software, que pudiese competir con productos similares del mercado, además de servir como experiencia de aprendizaje para el desarrollo de futuros proyectos de esta índole.
